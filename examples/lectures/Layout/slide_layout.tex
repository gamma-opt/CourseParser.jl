% Slide theme definitions
\usetheme{default}

\setbeamertemplate{navigation symbols}{}
\definecolor{blue}{rgb}{0.23,0.58,0.89}
\usecolortheme[rgb={0.23,0.58,0.89}]{structure}
\setbeamertemplate{itemize subitem}{--}

\setbeamercolor{alerted text}{fg=orange}
\setbeamertemplate{theorems}[numbered] 

\setbeamertemplate{footline} {
	\begin{beamercolorbox}[ht=2.5ex,dp=1.125ex,leftskip=.8cm,rightskip=.6cm]{structure}
		\footnotesize  Fabricio Oliveira
		\hfill
		\footnotesize \insertsection
		\hfill
		{\insertframenumber /\inserttotalframenumber}
	\end{beamercolorbox}
	\vskip 0.25cm
}

\AtBeginSection[] 
{ 
	\begin{frame}<beamer> 
		\frametitle{Outline of this lecture} 
		\tableofcontents[currentsection]  
	\end{frame} 
} 

\setlength{\parskip}{0.5em}

% Packages
\usepackage{
	graphicx,
	amsmath,
	amssymb,
	tikz,
	psfrag,
	color,
	blkarray,
	algorithm, 
	algorithmicx,
	algpseudocode,
	caption,
	colortbl,
	subcaption
}

\captionsetup[figure]{font=footnotesize, skip=0pt}

% Environment for theorems spanning two slides
\makeatletter
\newenvironment<>{longproof}[1][\proofname]{%
    \par
    \def\insertproofname{#1\@addpunct{.}}%
    \usebeamertemplate{proof begin}#2}
  {\usebeamertemplate{proof end}}
\makeatother

\newtheorem{proposition}{Proposition}[theorem]
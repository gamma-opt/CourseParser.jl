\paperheight = 29.70 cm  \paperwidth = 21.0 cm  \hoffset        = 0.46 cm
\headheight  =  0.81 cm  \textwidth  = 15.0 cm  \evensidemargin = 0.00 cm
\headsep     =  0.81 cm	 \textheight = 9.00 in  \oddsidemargin  = 0.00 cm					
\setlength{\parskip}{6pt}
\setlength{\parindent}{0pt} 

\usepackage{eurosym}
\usepackage{amsfonts}
\usepackage{amssymb}
\usepackage{amsmath}
\usepackage{enumitem}                       % for customizing the lists
\usepackage{graphicx}
\usepackage{tikz,pgfplots}
\usetikzlibrary{shapes}
\usepackage{color}
\usepackage{fancyhdr}
\usepackage{subcaption}
\usepackage{ntheorem}
\usepackage[utf8]{inputenc}
\usepackage[pdfpagemode  = None,
		    colorlinks   = true,
		    urlcolor     = blue,
            linkcolor    = black,
            citecolor    = black,
            pdfstartview = FitH]{hyperref}

\usepackage{apacite}
\usepackage{natbib}
\usepackage{booktabs}
\usepackage{float}
\usetikzlibrary{arrows}
\usepackage{enumitem}
\usepackage{multicol}
\usepackage{array}
\usepackage{varwidth}

\setlist{align=left,topsep=0pt,itemsep=-1ex,partopsep=1ex,parsep=1ex}
\setlist[enumerate]{topsep=0pt,leftmargin=*,labelsep=2ex,itemsep=-1ex}


% Generator functions

% Counters
\newcounter{homeworknumber}
\newcounter{exercisenumber}
\newcounter{problemnumber}
\newcommand{\HomeworkNumber}{\arabic{homeworknumber}}
\newcommand{\ExerciseNumber}{\arabic{exercisenumber}}
\newcommand{\ProblemNumber}{\arabic{problemnumber}}

% Arguments: #1: Exercise number
%			 #2: Date
%			 #3: Guidelines and links
\newcommand{\MakeExercise}[3]
{
	\clearpage 
	\addtocounter{exercisenumber}{#1}
	\setcounter{problemnumber}{0}
	\lhead{\bf MS-E2122 - Nonlinear optimization\\
               Prof. Fabricio Oliveira}
	\rhead{\bf Exercise sheet #1 \\ #2}
	\vspace*{-38pt}
	\noindent
	#3
}


% Arguments: #1: Homework number
%			 #2: Deadline
%			 #3: Guidelines and links
\newcommand{\MakeHomework}[4]
{
	\clearpage 
	\setcounter{homeworknumber}{#1}
	\setcounter{problemnumber}{0}
	\lhead{\bf MS-E2122 - Nonlinear optimization\\
               Prof. Fabricio Oliveira}
	\rhead{Homework #1 \\ \color{blue}\bf DEADLINE: #2}
	\vspace*{-38pt}
	\noindent
	#3
}


% Arguments: #1: Problem description (optional)
%			 #2: Problem text in a tex file
%			 #3: Where the problem will be presented (home/session)	 
\newcommand{\Problem}[3]
{
	\addtocounter{problemnumber}{1}
	
	\vspace{12pt}
	{\noindent \large \textbf{({#3}) Problem \ExerciseNumber.\ProblemNumber{}: #1}}
	\vspace{6pt}
	
	\noindent \input{#2}	
}


% Arguments: #1: Problem description (optional)
%			 #2: Problem text in a tex file
\newcommand{\HWProblem}[2]
{
	\addtocounter{problemnumber}{1}
	
	\vspace{12pt}
	{\noindent \large \textbf{Problem \HomeworkNumber.\ProblemNumber{}: #1}}
	\vspace{6pt}
	
	\noindent \input{#2}
}

\pagestyle{fancy}


% Theorem layout
\newtheorem{theorem}{Theorem}
\newtheorem*{theorem-non}{Theorem}
\newtheorem{proof}{Proof}
\newtheorem*{proof-non}{Proof}